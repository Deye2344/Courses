\chapter{接口设计}
\section{外部接口}

\subsection{账户接口}
账户的接口应该可以包括常见的电子账户,比如支付宝、微信、银行卡、Paypal等等,
在服务器内部形成统一的接口。需要实现的功能包括:电子账户添加、电子账户认证、
电子账户余额查询、电子账户充值到商店虚拟账户、商店虚拟账户转账到电子账户。

提供给内部的接口应该针对开发者管理系统和用户管理系统。

{\color{red}
	接口的具体定义为
	
\begin{lstlisting}[language=C, caption=账户接口, label={code:first-code}]
{
	string username //用户名
	string passwd //密码
	string account //账户
	vector<int> code //验证码
	float money //转账金额
	int error //错误码
}
\end{lstlisting}
}


\subsection{安装/卸载接口}
该接口用于客户端安装或卸载某个应用,只需提供该应用的安装/卸载程序的文件描述符或文件路径以及安装目录(如果是安装的话)即可。

{\color{red}
	接口的具体定义为
	
\begin{lstlisting}[language=C, caption=安装/卸载接口, label={code:first-code}]
{
	string username //用户名
	vector<int> code //验证码
	string file //文件路径
}	
\end{lstlisting}
}

\subsection{用户接口}
用户接口分为提供给普通用户和开发者的接口。

提供给普通用户的接口包括登录、注册、应用查询、应用购买、应用安装、应用删除、应用评分等等。

提供给开发者的接口包括登录、注册、应用上传、应用更新、应用删除、收入查询、转账提现、评论查看、评论回复等等。

分别参见表\ref{tab:developer_interface}和表\ref{tab:client_interface}。

\begin{longtable}{|p{4cm}|p{5cm}|p{5cm}|}
	\caption{开发者与应用商店系统的接口}\label{tab:developer_interface} \\
	\hline
	\textbf{开发者需求} & \textbf{输入} & \textbf{输出}\\
	\hline
	\endfirsthead
	\hline
	\textbf{开发者需求} & \textbf{输入} & \textbf{输出}\\
	\hline
	\endhead
	\hline 
	\endfoot
	\hline
	\endlastfoot
	登录 & 账号、密码 & 登录成功与否\\
	注册 & 账号、密码 & 注册成功与否\\
	应用上传 & 本地应用 & 上传成功与否\\
	应用更新 & 本地应用和待更新应用id & 更新成功与否\\
	应用删除 & 待删除应用id & 删除成功与否\\
    收入查询 & 无 & 收入\\
    转账提现 & 银行账号 & 转账或提现成功与否\\
	评论回复 & 回复内容 & 回复成功与否\\
	\end{longtable}

\begin{longtable}{|p{4cm}|p{5cm}|p{5cm}|}
	\caption{普通用户与应用商店系统的接口}\label{tab:client_interface} \\
	\hline
	\textbf{用户需求} & \textbf{输入} & \textbf{输出}\\
	\hline
	\endfirsthead
	\hline
	\textbf{用户需求} & \textbf{输入} & \textbf{输出}\\
	\hline
	\endhead
	\hline 
	\endfoot
	\hline
	\endlastfoot
	登录 & 账号、密码 & 登录成功与否\\
	注册 & 账号、密码 & 注册成功与否\\
	应用查询 & 应用id & 相关的应用,以列表呈现\\
	应用购买 & 待购买应用id和银行账户 & 购买成功与否\\
	应用安装 & 待安装应用id & 安装成功与否\\
    应用删除 & 待删除应用id & 删除成功与否\\
	应用评分 & 待评价应用id和评分 & 评分成功与否\\
	应用评价 & 待评价应用id和评论 & 评论成功与否\\
	\end{longtable}

用户接口最终将以图形界面呈现。


{\color{red}
	接口的具体定义为
	
\begin{lstlisting}[language=C, caption=安装/卸载接口, label={code:first-code}]
{
	string username //用户名
	string passwd //密码
	string command //指令
	string permission //权限等级
}
\end{lstlisting}
}

\section{内部接口}

\subsection{客户端}
如第三章的描述,客户端主要有这几个模块:注册或登录模块、普通用户应用管理模块、普通用户信息反馈模块、开发者应用管理模块、开发者信息反馈模块。

后面4个模块相对独立,相互之间没有交互。注册或登录模块会作为子模块被后面4个模块调用。


{\color{red}
	接口的具体定义为
	
\begin{lstlisting}[language=C, caption=安装/卸载接口, label={code:first-code}]
{
	vector<int> number //终端编码
	vector<int> code //验证码
	int command //指令编号
	data_type data //数据
}
\end{lstlisting}
}

\subsection{服务器端与数据库}
服务器端内部提供数据库统一接口,提供给开发者管理系统、应用管理系统、用户管理系统、第三方管理系统
使用,目的是提高安全性与简便。

提供高阶的增删改查接口,不需要直接使用PL/SQL.

不同的功能需要有不同的权限,比如修改删除需要的权限高于查询,同时,只有管理员才有直接使用PL/SQL查询
的权限。另外,应该提供方便的增加高阶功能的接口。


{\color{red}
	接口的具体定义为
	
\begin{lstlisting}[language=C, caption=安装/卸载接口, label={code:first-code}]
{
	vector<int> number //服务器编码
	vector<int> code //验证码
	int permission //权限等级
	string command //指令
	data_type data //数据
}
\end{lstlisting}
}