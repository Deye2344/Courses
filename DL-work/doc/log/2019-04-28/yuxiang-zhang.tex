\subsection{\ZhangYX}

\subsubsection{小组进展}

\begin{itemize}
\item 任正行
  \begin{itemize}
    \item 时间主要花在华为的比赛上,28号比赛结束
    \item 梳理大致的思路,主要参考popl13文章模型实现思路
    \item 没有书面总结,todo list和时间安排会在今晚发到群里
  \end{itemize}
\item 董恒
  \begin{itemize}
    \item 详细阅读上周组会中提到的相关工作,比较各工作的特点并记录了文档,包括RI、CPO、SRL、AutoRL等
    \item 工作量太大,需要整理下工作层次,增量实现
    \item 思路并不完善,暂时没有和陈广大进一步交流
  \end{itemize}
\end{itemize}

\subsubsection{个人进展}

\begin{itemize}
  \item 阅读\href{https://groups.csail.mit.edu/graphics/hdrnet/data/hdrnet.pdf}{Deep Bilateral Learning for Real-Time Image Enhancement\cite{gharbi2017deep}},补充工作动机、贡献和效果评估,详见\autoref{sec:gharbi2017deep}
  \item 了解Pytorch的算子扩展机制
  \begin{itemize}
    \item \href{https://pytorch.org/docs/master/notes/extending.html}{EXTENDING PYTORCH}
    \item \href{https://pytorch.org/tutorials/advanced/cpp_extension.html}{CUSTOM C++ AND CUDA EXTENSIONS}
    \item 尝试基于Pytorch扩展切片算子
  \end{itemize}
\end{itemize}

\subsubsection{工作计划}

\begin{itemize}
  \item 阅读Gradient Halide源码,查看是否有提供切片算子实现,尝试复现文章中的Evaluation
  \item 完成切片算子基于Pytorch提供的python扩展和C++/CUDA扩展实现
\end{itemize}