\PassOptionsToPackage{unicode=true}{hyperref} % options for packages loaded elsewhere
\PassOptionsToPackage{hyphens}{url}
%
\documentclass[]{article}
\usepackage{lmodern}
\usepackage{amssymb,amsmath}
\usepackage{ifxetex,ifluatex}
\usepackage{fixltx2e} % provides \textsubscript
\ifnum 0\ifxetex 1\fi\ifluatex 1\fi=0 % if pdftex
  \usepackage[T1]{fontenc}
  \usepackage[utf8]{inputenc}
  \usepackage{textcomp} % provides euro and other symbols
\else % if luatex or xelatex
  \usepackage{unicode-math}
  \defaultfontfeatures{Ligatures=TeX,Scale=MatchLowercase}
\fi
% use upquote if available, for straight quotes in verbatim environments
\IfFileExists{upquote.sty}{\usepackage{upquote}}{}
% use microtype if available
\IfFileExists{microtype.sty}{%
\usepackage[]{microtype}
\UseMicrotypeSet[protrusion]{basicmath} % disable protrusion for tt fonts
}{}
\IfFileExists{parskip.sty}{%
\usepackage{parskip}
}{% else
\setlength{\parindent}{0pt}
\setlength{\parskip}{6pt plus 2pt minus 1pt}
}
\usepackage{hyperref}
\hypersetup{
            pdfborder={0 0 0},
            breaklinks=true}
\urlstyle{same}  % don't use monospace font for urls
\setlength{\emergencystretch}{3em}  % prevent overfull lines
\providecommand{\tightlist}{%
  \setlength{\itemsep}{0pt}\setlength{\parskip}{0pt}}
\setcounter{secnumdepth}{0}
% Redefines (sub)paragraphs to behave more like sections
\ifx\paragraph\undefined\else
\let\oldparagraph\paragraph
\renewcommand{\paragraph}[1]{\oldparagraph{#1}\mbox{}}
\fi
\ifx\subparagraph\undefined\else
\let\oldsubparagraph\subparagraph
\renewcommand{\subparagraph}[1]{\oldsubparagraph{#1}\mbox{}}
\fi

% set default figure placement to htbp
\makeatletter
\def\fps@figure{htbp}
\makeatother


\date{}

\begin{document}

\hypertarget{header-n654}{%
\section{前言}\label{header-n654}}

析字真挺有意思的,特别是在整个学期很多内容都与析字相关,所以本着对析字知识做一个调研与总结以及谈谈自己的感悟的想法写下这篇小论文。参考文献部分将在文中注明。

\hypertarget{header-n223}{%
\section{析字分类}\label{header-n223}}

\hypertarget{header-n695}{%
\subsection{综述}\label{header-n695}}

首先,一个字有形、音、义三方面,所以大概想当然可以认为析字分为三个方面,化形、谐音、衍义{[}8{]}.{[}9{]}。分类如下:

\begin{quote}
\begin{itemize}
\item
  化形析字

  \begin{itemize}
  \item
    离合:离合字形
  \item
    增损:增损字形
  \item
    借形:单单假借字形
  \end{itemize}
\item
  谐音析字

  \begin{itemize}
  \item
    借音:单纯谐音
  \item
    切脚:利用反切上用做反切的两音
  \item
    双反:利用反切上顺倒双重反切
  \end{itemize}
\item
  衍义析字

  \begin{itemize}
  \item
    代换:换话达意
  \item
    牵连:随语牵涉
  \item
    演化:弯弯曲曲,演述得似有关联又似乎无关联,须细推究才能明白
  \end{itemize}
\end{itemize}
\end{quote}

其他地方{[}10{]},也给出了类似的分类。下面先一一解析。由于专业术语实在枯燥,不妨用案例给出,一目了然。

\hypertarget{header-n272}{%
\subsection{化形析字}\label{header-n272}}

化形析字可分为``离合''、``增损''、``借形''三类。

\hypertarget{header-n276}{%
\subsubsection{离合}\label{header-n276}}

将合体的字离析,或是将独体加以合并的,是为``离合''。

如《后汉书.五行志》中记载``\emph{千里草},何青青;\emph{十日卜},不得生。'',即是将``千、里、艸(草)''三字合为``董'',``十日卜''三字合为``卓'',暗喻东汉末年的权臣``董卓''。恰好我也姓董(与``董卓''无关),也析字为:``\emph{千里草},长青青;X
X X,\emph{亘古心}'';

\hypertarget{header-n278}{%
\subsubsection{增损}\label{header-n278}}

将一个字的字形增添或减少一部分笔划的析字法,称为``增损''。

如《镜花缘》第八十六回``\ldots{}王主,绰号叫做\emph{硬出头的}王大'',即是以``王''字头上加一点(出头)为``主''字的增损法起绰号。

\hypertarget{header-n456}{%
\subsubsection{借形}\label{header-n456}}

汉字中常有一字多音、一字多义的情形,在套用古语时故意将其中一两字改为别的音义,进而改变全句意思的,就称为``借形''。

如沈复《浮生六记》中所载,陈芸用《后赤壁赋》典语``今日之游乐乎?''提问,沈复友人亦以《左传.僖公三十年》典语``非\emph{夫人}之力不及此''回答。但在《左传》原文中,``夫''字应读为阳平声,为指示形容词,即``此''、``彼''之义,指秦穆公。此处则读为阴平声,将``夫人''借作``太太''之义,指沈复的夫人陈芸。

\hypertarget{header-n475}{%
\subsection{谐音析字}\label{header-n475}}

\hypertarget{header-n539}{%
\subsubsection{借音}\label{header-n539}}

两字字形不同,但因音相近而借用的,叫做``借音''。例如歇后语``姓郑的嫁给姓何的-正合适(郑何氏)''。

\hypertarget{header-n533}{%
\subsubsection{切脚 双反}\label{header-n533}}

没看懂。暂且不表述。

\hypertarget{header-n450}{%
\subsection{衍义析字}\label{header-n450}}

\hypertarget{header-n551}{%
\subsubsection{牵连}\label{header-n551}}

随着话中甲字的字义,故意牵扯出乙字来的方式。

例如《红楼梦》第二十八回中林黛玉说:``倘或明儿\emph{宝姑娘}来,什么\emph{贝姑娘}来,也得罪了,事情也就大了。''句中``宝姑娘''是对薛宝钗的习称,真有其人;而``贝姑娘''并无此人,只是依照平日``宝贝''二字常连用而牵附出来的。

\hypertarget{header-n583}{%
\subsubsection{演化}\label{header-n583}}

若隐藏本字,需由表面字义细细推求才可得出本义的方式,称为``演化''。

例如《红楼梦》第五回中的《金陵十二钗又副册》,以``\emph{霁月}难逢,\emph{彩云}易散''暗寓``晴雯''之名,就是演化法。

另外还有一种综合析字。就是同时使用两种以上的析字法。文中{[}10{]}提到的一个典故,但是我没看懂,粘贴如下:

\begin{quote}
《世说新语-捷悟》中记载,曹娥碑碑背题``\emph{黄绢幼妇外孙齑臼}'',就要先以衍义析字中的``演化'',将原八字代换为``色丝、少女、女子、受辛'',再以化形析字中的``离合''法,合为``绝(絶)妙好辞(辤)''四字。
\end{quote}

\hypertarget{header-n282}{%
\section{析字应用}\label{header-n282}}

\hypertarget{header-n636}{%
\subsection{现代}\label{header-n636}}

析字方式很多用于了学生的识字,比如黄{[}1{]}曾在小学的课堂上使用析字,``析形索义,善教会意字,保证有效识字''就是这种思想,例子如``炙'',上下结构,``火''烤``月''(指肉),表示很热。

更有人{[}5{]}.
{[}7{]}将析字法用于文言文的教学,使原本高考很枯燥的文言文变得灵活起来。文中将析字法用于三个方面,逐步深入,一是理解文言字词,二是把握文章底蕴,三是接受传统文化熏陶。仅仅举其一的一个例子:''除``字,《说文》里面说''除,殿陛也``,左边的双耳旁本意是阶梯或不是很高的山,而右边的''余``字形像远古社会人们用茅草房搭建的房屋,也就是''舍``。所以''除``本意指宫殿的台阶,后泛指台阶。又由于台阶需要经常打扫、清理落叶等异物,所以''除``引申为''清除,去掉``。还由于台阶也能表示人高升,所以还引申为``拜官、授予官职''。这样来解析,本来很难记的``除''的各种意思突然清晰起来,而且很难忘记。

甚至有人{[}6{]}将析字用于学习地理。虽然有些牵强附会,但是对于记忆却是十分有效的,比如文中举出的``疆''字,``弓''字代表牧业、武力,也就是古代的新疆主要是放牧人,需要弓箭防卫。``弓''下面的``土''表示帕米尔高原,右边的三横分别代表阿尔泰山、天山、昆仑山,而两个``田''字代表准格尔盆地和塔里木盆地。

\hypertarget{header-n290}{%
\subsection{古代}\label{header-n290}}

说完现代的析字用于教学和学习,下面说说几个古代的小故事。

傅{[}2{]}在一篇小文章里面提到几个有趣的小故事,是关于析字联的,比较明显的是一首戒赌诗:

\begin{quote}
贝者是人不是人,

只为今贝起祸根;

有朝一日分贝了,

到头成了贝戎人。
\end{quote}

也就是``赌一一贪一一贫一一贼``一条路。另一首联,就相当巧妙了:

\begin{quote}
寸土为寺,寺旁言诗,诗曰:明月送僧归古寺;

双木成林,林下示禁,禁云:斧斤以时入山林。
\end{quote}

上联中``寸''与``土''合成``寺''字,``寺''与``言''合为``诗''字;第二旬以``寺''开头,与前``寺''相连,用了``顶针'',用``诗''作后,第三句又用``诗''开头,再来个``顶针'',最后又用``寺''作结,与前面``寺''字照应;``明''字拆开去了``日''字,就留下``月''字。下联也是如此,颇为巧妙。

同样的另一篇小故事{[}3{]}就不那么容易看明白的,说的是一个书生被某佛寺拒之门外后留下的一首诗:

\begin{quote}
龛龙去东海,时日隐西斜。

敬文今不在,碎石入流沙。
\end{quote}

初看上去,没有什么特别,但是联系一下析字,就会看到,按每句诗拆字:龛去龙,时隐日,敬少文,碎走石。剩下''合寺苟卒``,谐音寓意就是''合寺狗足``。很有意思。

由于析字是通过字形结构离合分析从而使字义更加充实的一种表达技巧,在古诗词中,这一技巧常被用于调整语言或寓意托旨{[}4{]}。比如苏轼的《夜烧松明火》''坐看十八公,俯仰灰烬残``。其中的''十八公``,合起来就是''松``。又吴文英的《唐多令》''何处合成愁?离人心上秋``。''心上秋``就是''愁``。作者还举出了两个例子,其中一个是秦观的''水边灯火渐人行,天外一钩残月带三星``,后一句的''一钩``''三省``,作者解释说这是''心``,初看觉得有些牵强,但是看到这词的词名《赠陶心儿》,才觉得真是独具匠心。

\hypertarget{header-n331}{%
\section{感悟}\label{header-n331}}

以前觉得析字无非是古人在玩文字游戏,无聊至极,后来在课上某些章节,然后热别是对析字做了一些调研之后,越来越觉得部分的析字方法很巧妙,有一些案例也很有趣味。

\hypertarget{header-n686}{%
\subsection{遇到的问题}\label{header-n686}}

同时也遇到了一些问题,核心就是文化的脱节,比如读音的变化,一些析字现在根本看不出来;再比如繁体字和简体字的区别,这一点在上面的综合析字有体现,体现更多的是这篇博文{[}11{]}中列举的,都是用繁体字写出来的,对于非专业的人,读起来颇有难度。除了读起来的难度,更加麻烦的是化形析字的部分,如果化形的是繁简不同体,那么如果没有提示,根本就没办法看出来。

\end{document}
