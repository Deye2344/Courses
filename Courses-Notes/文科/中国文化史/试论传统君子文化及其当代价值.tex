\PassOptionsToPackage{unicode=true}{hyperref} % options for packages loaded elsewhere
\PassOptionsToPackage{hyphens}{url}
%
\documentclass[]{article}
\usepackage{lmodern}
\usepackage{amssymb,amsmath}
\usepackage{ifxetex,ifluatex}
\usepackage{fixltx2e} % provides \textsubscript
\ifnum 0\ifxetex 1\fi\ifluatex 1\fi=0 % if pdftex
  \usepackage[T1]{fontenc}
  \usepackage[utf8]{inputenc}
  \usepackage{textcomp} % provides euro and other symbols
\else % if luatex or xelatex
  \usepackage{unicode-math}
  \defaultfontfeatures{Ligatures=TeX,Scale=MatchLowercase}
\fi
% use upquote if available, for straight quotes in verbatim environments
\IfFileExists{upquote.sty}{\usepackage{upquote}}{}
% use microtype if available
\IfFileExists{microtype.sty}{%
\usepackage[]{microtype}
\UseMicrotypeSet[protrusion]{basicmath} % disable protrusion for tt fonts
}{}
\IfFileExists{parskip.sty}{%
\usepackage{parskip}
}{% else
\setlength{\parindent}{0pt}
\setlength{\parskip}{6pt plus 2pt minus 1pt}
}
\usepackage{hyperref}
\hypersetup{
            pdfborder={0 0 0},
            breaklinks=true}
\urlstyle{same}  % don't use monospace font for urls
\setlength{\emergencystretch}{3em}  % prevent overfull lines
\providecommand{\tightlist}{%
  \setlength{\itemsep}{0pt}\setlength{\parskip}{0pt}}
\setcounter{secnumdepth}{0}
% Redefines (sub)paragraphs to behave more like sections
\ifx\paragraph\undefined\else
\let\oldparagraph\paragraph
\renewcommand{\paragraph}[1]{\oldparagraph{#1}\mbox{}}
\fi
\ifx\subparagraph\undefined\else
\let\oldsubparagraph\subparagraph
\renewcommand{\subparagraph}[1]{\oldsubparagraph{#1}\mbox{}}
\fi

% set default figure placement to htbp
\makeatletter
\def\fps@figure{htbp}
\makeatother


\date{}

\begin{document}

\hypertarget{header-n0}{%
\section{摘要}\label{header-n0}}

从君子的由来谈起,到君子的品格、要求规范等等,再到君子的内涵,逐步深入,详细了解传统君子文化,最后结合具体案例解析其时代价值。最后得出,君子文化有它的价值,但是并不能解决实际问题的结论。

\hypertarget{header-n4}{%
\section{Abstract}\label{header-n4}}

We analyze the culture of Junzi from its origin to Junzi's characters,
norm and so on. Then gradually deepen into the connotation of Junzi, and
try to understand the culture of Junzi in detail. Next , we analyze its
value of the times with specific cases. Finally, it is concluded that
the culture of Junzi has its value, but it cannot solve any actual
problem.

\hypertarget{header-n7}{%
\section{绪论}\label{header-n7}}

君子文化是中国优秀的传统文化之一,固然有它理想主义的一面,但是它所蕴含的对社会的期待却能反映出中国人几千年来对社会中人的约束和期待,所以探究君子文化以及对现实的意义还是很有必要的。

本文按照如下的逻辑结构,首先结合文献和个人理解探究君子文化的含义,这一过程是逐步深入的,然后根据现代的发展谈谈君子在现代应该如何看待。在如何看待的问题上,尽可能给出比较具体的想法和案例。

\hypertarget{header-n12}{%
\section{``君子''由来及其简短定义}\label{header-n12}}

\hypertarget{header-n13}{%
\subsection{``君子''由来}\label{header-n13}}

在维基百科{[}1{]}中给出了非常简短但是明确的说明

\begin{quote}
君子在中国周朝之前,为贵族的统称。后来在春秋的封建社会则变成了士大夫的统称,也就是为官之人称为君子,平民称为小人。后来,儒家认为君子应该不单单指贵族或士大夫,而是``圣人之下,富有礼仪规范的人'',具有高道德标准的人
\end{quote}

在{[}5{]}中有更加明确的划分,即按照从西周到孔子、百家争鸣时期、西汉王朝之后划分为三个阶段,可以看到,儒家将君子从一个和地位相关的专有词,变成了仅仅和品行相关的任何人只要满足了要求都能使用的词。更进一步,儒家把君子看成是塑造自己人格理想的对象,是人生的终极目标。在传统的儒家文化里,有很多的君子的相关标准,比如说,君子有所为,有所不为;舍身取义;士不可以不弘毅;君子爱财取之以道等等;从这一层面来说,如果社会上人人都是君子,可真的就是大同盛世了。

\hypertarget{header-n21}{%
\subsection{简短定义}\label{header-n21}}

首先,得先知道君子是什么,是怎么定义的,然后才能判断,这在当代有什么价值

用一句简练的话来概括君子,其实就是``君子进德修业''(《易传》),也就是说{[}8{]}君子之道可以简单地概括为两个方面,一是不断地增进自己的德行,二是卓然挺立自己的事业。

\hypertarget{header-n26}{%
\section{``君子''品格及其时代价值}\label{header-n26}}

\hypertarget{header-n27}{%
\subsection{``君子''品格}\label{header-n27}}

仅仅简要说明君子的品格是不够的,需要有更加全面的概括和认识,在{[}2{]}中是从君子具备的品格来分析的,作者认为,君子应该具有的品格有:

\begin{enumerate}
\def\labelenumi{\arabic{enumi}.}
\item
  \textbf{自强}。也就是自立和奋进图强,不能啃老、懒散或者依赖他人救济。正如易经中所说,``天行健,君子以自强不息''。
\item
  \textbf{慎独}。这个词即便是在现代,也被很多的自我约束能力较高的人频繁使用。不同于西方的上帝无处不在的监督,在中国,没有任何人的监督,都要做到表里如一,人前人后都一个样。正如孔子所说:``君子慎其独''。
\item
  \textbf{克己}。人是充满私欲的,而且不断膨胀,所以需要自己的克制,同时克己也是在克制自己的弱点,比如惰性、懦弱、情绪等等。孔子也提倡``克己复礼'',也就是克制自己,是自己的言行符合规范。
\item
  \textbf{宽人}。宽以待人,严于律己。这也正是孔子一直强调的``自所不欲,勿施于人''。首先要自己做到,而别人做不做得到并没有什么关系。
\item
  \textbf{诚信}。所谓``君子一言,驷马难追''。诚信可以理解为真诚和守信,这一点在非常缺失的当代具有很大的意义。
\item
  \textbf{谦虚}。谦虚应当是中国人独有的一种品质,直到现在都一直溶于骨血里,中国人认为``谦受益,满招损'',就是说不要自满、骄傲、自负等等。当然这一点是否真的需要每个人都做到还有待商榷。
\item
  \textbf{忠}
  。忠意味着爱国,尽心尽力。如何表达忠,大概也是个值得考虑的问题,毕竟存在着大量的愚忠,关于如何爱国似乎也有更多的讨论。
\end{enumerate}

\hypertarget{header-n52}{%
\subsection{时代价值}\label{header-n52}}

仅仅从上面的这几点来看,其中自强、慎独、克己、诚信都是争议不大的即便是现代人也应该遵守的。

而其他的,比如说,宽人,有很大的争议,比较流行的一种说法是,中国人总是苛求好人,放纵坏人。甚至有一种曲解为,好人需要九九八十一难才能成佛,而坏人只要放下屠刀立地成佛。不考虑这样的极端情况,仅仅说明普通的情况也的确是这样,微博上面家暴的、弑母的、吸毒的不断有人原谅,但是有的公众号发了篇医学辟谣的文章就很多人批判它不够严谨。又比如,谦虚,谦虚导致的一个最大问题就是,接受了谦虚教育的学生普遍自卑,不爱提出问题,这一点在尊师重道的古代似乎没有问题,但是在提倡学风自由的现代却显示出了它巨大的不足,但是反过来,一个自负骄傲的人,势必惹人厌,所以这个存在争议,只能说不能孤傲,但不能说,时刻谦虚。再比如,忠,这个忠应该针对谁?肯定不是君主,是不是国家呢?如果是,应该是哪个国家呢?这个也很难讲。如果把忠理解为尽心做事倒是无可非议。

\hypertarget{header-n57}{%
\section{``君子''内涵及其时代价值}\label{header-n57}}

\hypertarget{header-n58}{%
\subsection{``君子''内涵}\label{header-n58}}

上面简要分析了古代君子所要求的品格在现代是否可行的问题,总结来看,部分可行,部分有争议。

仅仅使用表面形容的几个词来理解君子的含义是不够的,透过现象看本质,需要查阅更多的资料,陈{[}3{]}曾指出使用现代文化来透视君子的人格,他将君子人格概括为4个方面:

\begin{enumerate}
\def\labelenumi{\arabic{enumi}.}
\item
  \textbf{复礼}。这里的礼指周朝的礼仪制度,孔子认为,可以通过礼来培养个体的道德人格。也就是通过对外在的约束来潜移默化地使人的内在合乎道德。
\item
  \textbf{仁}。仁是一个很复杂的概念,即是一种道德品质又是一种政治理想。在这里也一言两句说不清楚,就如同``道''在老子的概念里一样。具体可以参见维基百科{[}4{]}.
\item
  \textbf{重义贵和}。通过这种方式来达到人际间关系的和谐稳定。也就是``君子喻于义,小人喻于利''。
\item
  \textbf{中、和}。``中''也就是掌握一个度,过犹不及的意思,而``和''值平衡和谐。另一种说法是中庸之道。
\end{enumerate}

在这一种很抽象的对君子的诠释,概括性非常强,前面一种比较浅显的解释仅仅是说君子是个人的要求和约束,到了更深的层次才能发现,君子的概念与国家和社会的发展是紧紧结合在一起的,孔子对于君子概念的形成可以也是顺应了某种社会形式,所以他才认为这种方式可行。

\hypertarget{header-n78}{%
\subsection{时代价值}\label{header-n78}}

这里提到,君子其实是和社会的发展紧密相关的,所以有研究人员{[}9{]}指出,君子文化的当代价值应当分为三个层面

\begin{enumerate}
\def\labelenumi{\arabic{enumi}.}
\item
  \textbf{国家层面}。在这一层面上其实也就是天下兴亡,匹夫有责;鞠躬尽瘁,死而后已;舍身取义杀身成仁等等浩然正气。
\item
  \textbf{社会层面}。这也就是每个人都履行自己的社会职责,减小社会矛盾。
\item
  \textbf{个人层面}。也就是上面提到的很多个人规范。
\end{enumerate}

对于这三个层面的建设,我是反对的,这种说法其实仅仅具有口号的意思,并没有多大的建设价值,人与人之间的关系是十分复杂的,并不能简单地说,只要人人都XXX,世界就会XXX,这种没有天然社会动力的想法是很单纯的,在没有强大的支持下,根本无法进行。我对社会的粗浅认识是,社会其实是一种纳什均衡{[}10{]},
也就是说,现在达到的规范,如果有人不遵
守规范,自己的利益就会受损。但是君子的要求就不同了,他天然地要求自己的奉献,强调奉献没错,但是如果没有天然的动力来支撑,就没法继续让更多人遵守,就比如说,不君子的人,总是占便宜,君子总是吃亏,那么谁又愿意成为君子呢?这一点就如子贡解救了一名鲁国的奴隶而不要报酬,孔子却批评他一样,首先是有好处的,人们才愿意去做。一味地逆着宣传美好品德是没有用处的。这篇文章{[}11{]}更是表达了相同的观点,传统文化受到广泛重视,新儒家闪亮登场,但往往囿于训诂诠释,不能很好地联系实际、针砭时弊,往往与广大群众相脱离,悬在半空不接地气。

但是是有更加实际的、缓和的时代价值,这一点在下面解释。

\hypertarget{header-n95}{%
\section{``君子''可行吗?}\label{header-n95}}

\hypertarget{header-n96}{%
\subsection{缓和的概念}\label{header-n96}}

君子概念对人的要求是很高的,那么现实中是否能达到呢?洪{[}6{]}曾提到,儒释道三家都有理想人格,儒家的圣人,道家的神仙,佛家的佛,这三类都是完美无缺的人格,是人们向往的对象,几乎是不能达到的。而君子概念就不一样了,普通人在日常生活中努力去做也能达到,而不需要任何的天赋。在这种意义上,``君子
''文化在鼓励每个人努力向上、积极向善的方面,具有重要的积极意义
。所以说,君子文化的精髓不在于各种有些落后的个人规范形式,而在于向上向善的价值观人生观{[}5{]}。

程{[}12{]}提出了一种同样缓和,宽泛的时代价值,与上面的很像,主要是分成三个部分,一是家国情怀。即修身、齐家、治国、平天下。二是道德遵循。即仁义礼智信。三是人格力量。也就是人格影响、典范。这种说法是比较合理的,重要的不是几千年来各种可能过时的陈旧规范,而是君子背后蕴含的向上的力量。

朱{[}7{]}更是认为,对于高校的学生,培养君子人格,要承认每个人的不完美,重点在于激发学生内心对君子人格的向往,逐步引导学生不断审视自我,在学习、践行、自省、调整的过程中不断趋近于君子人格。

\hypertarget{header-n103}{%
\subsection{重提``君子''文化的原由}\label{header-n103}}

现在重提各类中国古代文化,特别是儒家的文化,原因我自己总结无非自己以下几种:

\begin{enumerate}
\def\labelenumi{\arabic{enumi}.}
\item
  文化不自信。科技原创内容比较低,就会取寻找过往的内容来填补。
\item
  尊古。总是认为古人的东西是最好的,而不承认时代是发展的。
\item
  社会的乱象。不得不说出现了很多奇怪的东西,比如12岁小孩弑母。似乎这一点在尊崇孝悌的古代不会发生,但是真的不会发生吗?
\end{enumerate}

\hypertarget{header-n116}{%
\section{''君子``文化时代价值}\label{header-n116}}

\hypertarget{header-n117}{%
\subsection{''君子``规范的落后}\label{header-n117}}

中国文化的精髓,不在于那些早已落后的概念、约束(比如诸多弟子规),而在于中国人始终积极向善、勤劳勇敢,不依赖于神(比如发洪水,西方是上帝用诺亚方舟救人,而中国是大禹治水)的优秀品格。如果去追求古代的规章制度、礼仪典范,可以说就是舍本逐末、买椟还珠了。

我们不能使用君子这样几千年都没有成功过的概念来要求现代的人,它有它好的一方面,就是看起来很美好,如果真的有君子社会,世界真的就完美了,但是这是绝对做不到的,几千年来从来都没有做到的事,虽然现在不能说现代人就做不到古代人做不到的事,但是有更加可行的规章制度的办法来约束人。

\hypertarget{header-n122}{%
\subsection{案例诠释}\label{header-n122}}

现在普遍出现的一个非常鼓舞人的现象是:毒品人人喊打。任何艺人、明星,如果他出轨,有人原谅,它偷税漏税,也有人原谅,但是如果吸毒,必将全网黑。这是古代文化能给予的吗?吸毒使人憔悴,但是为什么清末却是无可奈何,阻止不了?要知道那时候仍是儒家文化占领的国家。现在却能这么坚决,是因为有近代史的屈辱。而不是所谓的礼仪之邦的要求。

另一个比较残酷的案例是,12岁小孩弑母的事件。众人只看到了小孩的残酷,但是本质却是双方都很悲哀。家庭教育的缺失、经济实力的不足不是键盘侠几点传统文化的丢失就能解决的。

还有各种明星出轨的案例,这用古代文化,就更加没办法解决,在那个一夫多妻的时代,何况女子地位很低这种糟粕怎能重现?

其他的案例,比如题目给出的信息。桔子、货车翻倒被哄抢,这根本就不是''君子``文化教育的缺失,不是任何古代文化的缺失,而是教育的缺失,任何读过大学的,受过基本教育的、知道基本廉耻的人都不会去做。比如未成年犯罪,同样也是教育的缺失,让未成年人整天去背诵弟子规不会有任何改变。

\hypertarget{header-n131}{%
\subsection{"君子"文化没有价值了吗?}\label{header-n131}}

价值是有的,但是就文化层面而言,我们有更加美的诗词文不去学,反而在充满糟粕的儒家文化里寻找精华,就有点说不过去了。就个人层面来说,提倡''君子``文化无可厚非,但是其本质不是追寻古代的礼仪制度,而是在套用这样传统的概念来构建更加现代的社会。无需舍身取义,只要简单地遵守现代的基本礼仪。

最后,我们总说,取其精华去其糟粕,但是怀着胡适的心,就做不了鲁迅的事。

\hypertarget{header-n136}{%
\section{总结}\label{header-n136}}

本文逐步深入''君子``的概念,并且在不同层面提出相关的时代价值,然后整体来理解其时代价值。可以看到的是,''君子``文化是发展的,不是一成不变的,对它的重新提出也是某种时代的发展。

\hypertarget{header-n139}{%
\section{参考文献}\label{header-n139}}

{[}1{]} 维基百科{[}W{]}.
https://zh.wikipedia.org/wiki/\%E5\%90\%9B\%E5\%AD\%90

{[}2{]} 史少博. 论中国传统文化中的君子品格{[}J{]}. 社科纵横 SOCIAL
SCIENCES REVIEW: Jul,2018 VOL. 33 NO. 7

{[}3{]} 陈德峰. `` 君子 '' 人格的现代文化透视{[}J{]}. 浙江师范大学学报 (
社会科学版 ): No.3,2002, General No.119, Vol.27

{[}4{]}维基百科{[}W{]}. https://zh.wikipedia.org/wiki/\%E4\%BB\%81

{[}5{]}周玉清 王少安. 中华传统君子文化的历史发展及其当代价值{[}N{]}.
光明日报/2016 年/4 月/22 日/第 001 版

{[}6{]}洪修平. 中国文化中的君子与理想人格{[}N{]}.

{[}7{]}朱小芳. 中国传统``君子人格''对高校人格培育的当代价值{[}N{]}.

{[}8{]}王林伟. 君子之道与现代生活------传统道德的现代转化侧论{[}N{]}.
贵州大学报: Vol. 35, No. 2, Apr. 2017

{[}9{]}李 潇. 君子文化探析{[}N{]}. 湖北函授大学学报( 2017) 第 30 卷第 5
期 总第 195 期

{[}10{]}维基百科{[}W{]}.
https://zh.wikipedia.org/zh-hans/\%E7\%B4\%8D\%E4\%BB\%80\%E5\%9D\%87\%E8\%A1\%A1\%E9\%BB\%9E

{[}11{]}冰火. 君子文化的当代价值和现实意义{[}N{]}. 东方烟草报/2014 年/6
月/4 日/第 001 版

{[}12{]}程碧英. 论君子文化的时代内涵{[}N{]}. 成都大学学报( 社会科学版 ):
Serial No. 179, No. 5, Oct. 2018

{[}13{]}作者. 篇名{[}J{]}. 刊名,出版年份,期号:起止页码

\end{document}
