\subsection{Thread}
Julia 中对于 thread 的定义主要在 \emph{threading.h} 和 \emph{threadgroup.h} 中,对于 thread 的操作主要在 \emph{threading.c} 中。

首先 Julia 是根据当前操作系统去判断使用的 thread 操作的 API,对于 Windows,Julia 采用的主要是微软自家的线程操作,调用的是头文件 \emph{Windows.h} ;对于非 Windows 系统,Julia 采用的是 pthreads 库进行线程的操作。

\subsubsection{Thread-local storage (TLS)}

在 threading.c 中定义了一系列的 thread 操作,如初始化 thread,执行分配到 thread 上的函数等操作,Julia 在对线程进行这些操作时会用到 tls\_states buffer (Thread-local storage,Julia 的优化定义在 \emph{src/tls.h} 中),值得注意的是 Julia 对 tls 的使用也做了一些优化:

首先是因为 Mac 和 Windows 不是用 ELF,所以采用运行时的 API 来创建 TLS,据他们说这种方式要比直接使用 \emph{\_\_thread} 关键字的效率更高一些。

\begin{itemize}
    \item 对于 Mac:因为没有静态 TLS 模型,所以采用的是调用 pthread 库中的 API 来进行创建;
    \item 对于 Windows:则是使用微软自家的 \emph{TLSAlloc} 实现;
    \item 对于 Linux,FreeBSD:使用 \emph{\_\_thread} 关键字进行实现;
\end{itemize}

Julia 中具体的 TLS 结构在 \emph{src/julia\_threads.h} 中定义为 \emph{\_jl\_tls\_states\_t} 。

\subsubsection{Thread init}

对于 master thread 的初始化,Julia 同样根据操作系统进行了区分,对于 Windows 采用了 \emph{DuplicateHandle} 进行处理,而其他的操作系统,则由 \emph{ti\_initthread(0)} 进行。

对于其他线程,均是由 \emph{static void ti\_initthread(int16\_t tid)} 进行初始化,在该函数里面对目标线程的 tls\_states 进行了各项初始化:
\begin{lstlisting}
static void ti_initthread(int16_t tid)
{
    jl_ptls_t ptls = jl_get_ptls_states();
#ifndef _OS_WINDOWS_
    ptls->system_id = pthread_self();
#endif
    assert(ptls->world_age == 0);
    ptls->world_age = 1; // OK to run Julia code on this thread
    ptls->tid = tid;
    ptls->pgcstack = NULL;
    ptls->gc_state = 0; // GC unsafe
    // Conditionally initialize the safepoint address. See comment in
    // `safepoint.c`
    if (tid == 0) {
        ptls->safepoint = (size_t*)(jl_safepoint_pages + jl_page_size);
    }
    else {
        ptls->safepoint = (size_t*)(jl_safepoint_pages + jl_page_size * 2 +
                                    sizeof(size_t));
    }
    ptls->defer_signal = 0;
    void *bt_data = malloc(sizeof(uintptr_t) * (JL_MAX_BT_SIZE + 1));
    if (bt_data == NULL) {
        jl_printf(JL_STDERR, "could not allocate backtrace buffer\n");
        gc_debug_critical_error();
        abort();
    }
    memset(bt_data, 0, sizeof(uintptr_t) * (JL_MAX_BT_SIZE + 1));
    ptls->bt_data = (uintptr_t*)bt_data;
    ptls->sig_exception = NULL;
    ptls->previous_exception = NULL;
#ifdef _OS_WINDOWS_
    ptls->needs_resetstkoflw = 0;
#endif
    jl_init_thread_heap(ptls);
    jl_install_thread_signal_handler(ptls);

    jl_all_tls_states[tid] = ptls;
} 
\end{lstlisting}

\subsubsection{Thread function}
在 \emph{threading.c} 中还定义了处理分配到 thread 上函数的操作,主要定义在函数 \emph{void ti\_threadfun(void *arg)} 中,这里会首先创建一个新的 thread,然后对该线程进行一个栈空间的初始化,这个栈空间的初始化同样根据不同的操作系统采用的不同的栈分配 API(非 Windows 采用的 pthread),之后则是在分配好的栈空间上创建一个根任务,然后将该 thread 添加到 threadgroup 中初始化,从 threadgroup 的结构体中可以看到采用 libuv 库中的一些接口:
\begin{lstlisting}
typedef struct {
    int16_t *tid_map, num_threads, added_threads;
    uint8_t num_sockets, num_cores, num_threads_per_core;

    // fork/join/barrier
    uint8_t group_sense; // Written only by master thread
    ti_thread_sense_t **thread_sense;
    void              *envelope;

    // to let threads sleep
    uv_mutex_t alarm_lock;
    uv_cond_t  alarm;
    uint64_t   sleep_threshold;
} ti_threadgroup_t;
\end{lstlisting}
之后则会进入一个 \emph{for(::)} 循环来执行调用的任务,是对分配给当前线程上的任务进行处理,判断结构体 \emph{ti\_threadwork\_t work} 是否为空,根据 work 中的成员进行相关函数的的调用,以及根据成员状态判断该任务是否完成执行跳出循环。
\begin{lstlisting}
typedef struct {
    uint8_t command;
    jl_method_instance_t *mfunc;
    jl_callptr_t fptr;
    jl_value_t **args;
    uint32_t nargs;
    jl_value_t *ret;
    size_t world_age;
} ti_threadwork_t;
\end{lstlisting}