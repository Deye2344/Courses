\newcommand\showchange{TT}%{TT FL}%
\newcommand\hasalgo{FL}%{TT FL}%
%\documentclass[preprint]{../../share/sigplanconf}
\documentclass[a4paper]{article}

\input local    % common local setting
\input clocal   % common Chinese setting
\input people
\begin{document}

\title{Julia语言及其对新兴机器学习应用的支持探索}
\author{
\rm 张 昱 (yuzhang@ustc.edu.cn) 课题组\\
中国科学技术大学 计算机科学与技术学院
}

\maketitle
% turn on page numbers
%\pagestyle{empty}
\pagenumbering{arabic}
\newpage
本文介绍Julia语言特征及其编译运行机制、扩展包及其构造机制、
新兴机器学习应用需求以及现有解决对策(如Ray~\cite{moritz2018ray}),探讨如何用Julia解决新兴机器学习应用带来的一些挑战。
\subimport{./}{logs}
\newpage
\subimport{./}{resources}
\subimport{lang/}{part}
\subimport{compiler/}{part}
\subimport{packages/}{part}
\subimport{rayRL/}{part}
\subimport{papers/}{part}
\newpage
\begin{small}
\bibliography{pa,langs}
\bibliographystyle{plain}
\end{small}

\end{document}
