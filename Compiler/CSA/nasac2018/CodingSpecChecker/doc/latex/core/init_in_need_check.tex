\subsection{按需初始化检测}
\label{sec:core:init}

按需初始化检测本质上关注的是避免对内存区域的冗余写操作,
即,如果对一内存区域的连续两次访问操作均为写操作,
那么前一次写操作就是冗余的。

本工具采用基于符号执行的路径敏感的过程内分析完成该类检测。
在分析过程中,本工具追踪的程序点状态定义如\autoref{equ:init_state}所示,
意为各内存区域(Memory Region,简记$MR$)的上一次访问操作类型,
$R$ 表示读操作,$W$ 表示写操作。

\begin{equation}
\label{equ:init_state}
S : MRs \mapsto \{ R, W \}
\end{equation}

在上述程序状态基础上,转换规则和检查动作定义如\autoref{fig:init_check} 所示。
其中函数 $Value()$ 接收一个左值并返回对应右值。

\begin{figure}[htbp]
\begin{tabular}{l}
$[[entry]]:$\\
$\qquad S = \phi;$\\
$[[read\ Region]]:$\\
$\qquad S[Region \mapsto R];$\\
$[[write\ Region]]:$\\
$\qquad \text{if}\ S(Region)=W,\text{then report reduntant write}; S[Region\mapsto W];$\\
$Post.[[E(E_1,E_2,...,E_n)]]:$\\
$\qquad \forall E_i \in \{E_1,E_2,...,E_n\} \land IsLValue(E_i),
S[Value(E_i) \mapsto IsOutput(E, E_i)\ ?\ W : R]$
\end{tabular}
\caption{按需初始化检查状态转移与动作检查} \label{fig:init_check}
\end{figure}

% \begin{flalign*}
% &[[entry]]: \\
% &\qquad S = \phi; \\
% &[[read\ Region]]: \\
% &\qquad S[Region \mapsto R]; \\
% &[[write\ Region]]: \\
% &\qquad \text{if}\ S(Region)=W,\text{then report reduntant write}; S[Region\mapsto W] \\
% &Post.[[E(E_1,E_2,...,E_n)]]: \\
% &\qquad \forall E_i \in \{E_1,E_2,...,E_n\} \land IsLValue(E_i),
% S[Value(E_i) \mapsto IsOutput(E, E_i)\ ?\ W : R]
% \end{flalign*}
